\chapter{Discussion}
% "results indicate that..."
  Examining the trajectories of each of the pigeon models' heads as seen in \ref{fig:head_bs_0}, the resulting behavior exhibited by the pigeon model with a fixed body, whose deep reinforcement learning controller was trained on $r_{fifty\_fifty}$, indicate that the combination of visual stabilization and motion parallax are sufficient to generate a behavior that fixes the position of the head, resembling the hold phase in pigeons.

  On the other hand, the resulting behavior exhibited by the pigeon model with body speed 1, whose deep reinforcement learning controller was trained on the same reward function as seen in Figures \ref{fig:head_bs_1_all_trimmed} and \ref{fig:head_bs_1_closest_trimmed}, indicate that the 2 functionalities are insufficient to produce head-bobbing behaviors in moving bodies during forward locomotion, as he pigeon model did not need to periodically thrust its head forward to maximize Davies' equation depicting motion parallax between objects.

  Such results indicate that visual stabilization with retinal cells capable of detecting movement in all directions is enough to maximize the sum of external objects' angular velocities within the retina.
    Examining Davies' equation \ref{equ:davies_motion_parallax}, it can be hypothesized that reinforcement learning controllers trained on reward function that reflect such function would only reproduce head-bobbing behaviors when all objects within the retina are globally static, since the equation does not account for the objects' velocities.
    Since the external objects in our experiment had objects moving forwards and backwards, such may have automatically increased the reward function depicting Davies' equation of motion parallax per timestep.


% fifty_fifty
  One possible argument against our claim can be pointed to our decision to weigh $r_{head\_stabilize}$ and $r_{motion\_parallax}$ equally for the reward that the policy or controller representing preliminary hypotheses $r_{fifty\_fifty}$. One could argue that by changing the weights, it may be possible to produce different behaviors than those shown in the results.
  Specifically, by assigning a larger weight value for $r_{head\_stabilize}$ than that for $r_{motion\_parallax}$, one could expect a pigeon model with the body speed of 1 to reflect the task of constricting the global position of its head and produce hold phases while occasionally moving the head to induce motion parallax.
  However, considering that the maximum return for $r_{head\_stabilize}$ is 0 and the pigeon whose controller was trained on $r_{fifty\_fifty}$ still managed to produce high positive returns as seen in Figure \ref{fig:learning_rate_fifty_fifty_bs_1} , it can be inferred that penalizing movement of the head would not change the resulting behavior.
  Therefore, factors other than static visual stabilization $r_{head\_stabilize}$ and motion parallax induction $r_{motion\_parallax}$ are most likely also contributing to the induction of head-bobbing behaviors.

% behavior in bs=1; fifty_fifty
  % We examine the biggest behavioral difference between pigeon models in the baseline experiments and their counterparts reflecting preliminary hypotheses, particularly, those trained on $r_{fifty\_fifty}$ with the body speed of 1.
  % The latter example, unlike the former, presents a behavior where the head is moved downwards throughout the duration of the experiment (Figure \ref{fig:fifty_fifty_body_speed_1}).
  % Despite the major differences between the two experiments, the latter does not suffer from an inability to acquire rewards (Figure \ref{fig:learning_rate_fifty_fifty_bs_1}).
  % lack of muscular strain penalty?
  % needs higher details, such as an addition of muscular physics, and progression in incremental modeling
  % muscular simulation and their placements upon the skeletal model may lead to more stabilization (cite Geijtenbeek)

% may need a state machine-like mechanism for replication of head-bob
  % composed of the "hold phase" state and the "thrust phase" state
% hierarchical control system
  % hierarchical reinforcement learning should be used for modeling the control system with this hypothesis
  % examples of this working (DIAYN; Atari difficult game w FALCON)
% indicates that a hierarchical control system is embedded in pigeons' neurology.
% pattern generating module or functionality, such as central pattern generators seen in the spinal cortex

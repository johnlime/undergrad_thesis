\chapter{Discussion}
% The results indicate that
  % regarding the behavior exhibited by the pigeon model with body speed 1 whose deep reinforcement learning controller was trained on $r_{fifty\_fifty}$
  % while the combination of visual stabilization and motion parallax are sufficient to generate a behavior where the pigeon's head is fixed in one position
  % insufficient to produce head-bobbing behaviors in moving bodies during forward locomotion

  % did not need to constantly thrust head forward to maximize Davies' equation depicting motion parallax between objects
    % visual stabilization with retinal cells capable of detecting movement in all directions is enough to maximize the sum of external objects' angular velocities within the retina
    % Davies' equation only works when all objects within the retina are globally static and does not account for moving objects.
    % Since the external objects in our experiment had objects moving forwards and backwards, such may have automatically increased the reward function depicting Davies' equation of motion parallax per timestep.


% may need a state machine-like mechanism for replication of head-bob
  % composed of the "hold phase" state and the "thrust phase" state
% hierarchical control system
  % hierarchical reinforcement learning should be used for modeling the control system with this hypothesis
% indicates that a hierarchical control system is embedded in pigeons' neurology.
% pattern generating module or functionality, such as central pattern generators seen in the spinal cortex

% head goes downwards during forward locomotion
% lack of muscular strain penalty?

% needs higher details, such as an addition of muscular physics, and progression in incremental modeling
% muscular simulation and their placements upon the skeletal model may lead to more stabilization (cite Geijtenbeek)

\chapter{Approach}
% def of pigeon model
  We define a simplified 2-dimensional model of pigeons based on incremental modeling. The pigeon model consists of 3 joints connecting one body representing the head, 2 bodies representing the neck, and one body representing the torso. The model's physics, mainly the collision and gravity, is simulated in a 2 dimensional physics engine. The torso's orientation and y-position is fixed while its x-position is incremented by a constant value. This represents forward locomotion at a constant speed.

  Additionally, we build control systems for the model using reinforcement learning. By using reinforcement learning, we can train the controller to maximize reward functions that represent hypotheses or manually-defined trajectories for the bodies in the model to follow.

  As a baseline for the model and control system, we attempt to recreate the head-bob movement by setting a target position for the head's position to match every timestep. The target position is first defined at a set location in front of the pigeon model $T$ relative to the position of its torso. The target then acts as a static position in the global coordinate for the head to follow. If the distance between the target position and the torso's position goes below a set threshold value, the target is repositioned at the same location $T$ relative to the torso's position.

% Hypothesis Testing
% Hold phase: Head stabilization

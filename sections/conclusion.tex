\chapter{Conclusion} \label{ch:conclusion}
% In our research, we presented a simplified model of the head, neck, and body of pigeons
% built control systems that represent preliminary hypotheses on the functionalities of head-bobbing, specifically visual stabilization and induction of motion parallax
% compared the trajectory of the heads generated by such control systems to those generated by the baseline control system,
% which has the head follow manually-defined head-bobbing trajectories.
% The results showed that, while the pigeon with a stationary body, whose controller
% was defined by the preliminary hypotheses, generated a trajectory of the head that resembled its baseline counterpart,
% the pigeon with a moving body failed to do such.
% Therefore, we determined that the functionalities of head-bobbing proposed by preliminary hypotheses are not sufficient to reproduce such behaviors, and concluded that other unseen mechanisms, such as muscle strain and hierarchical control systems, may be essential in inducing the behavior.

% Future works should focus on proposing more detailed models of pigeons to progress the process of incremental modeling.
% It is recommended that such details include muscular dynamics to simulate muscle strain.
% Additionally, the control system for the pigeon model should be hierarchical, where the high-level controller determines whether the low-level controller should induce the thrust or hold phase.
% Such can be accomplished by utilizing algorithms such as hierarchical reinforcement learning.

\chapter{Introduction}
Biomimetics is a research field that attempts to replicate biological phenomena, such as behaviors exhibited by animals, using methods seen in engineering.
Models developed in biomimetics can be applied to development of bio-inspired robotics, which can help solve difficult engineering problems similar to those solved by biological organisms.
  For example, engineers building control systems for bipedal locomotion robots can use proposed models of bipedal locomotion in bipedal organisms, since such organisms have already solved problems that would arise while attempting to build such control systems, such as preventing the body from falling over during the said locomotion.

Head-bobbing is a behavior unique to small birds, mainly pigeons, which consists of stabilizing their heads in a single location while occasionally altering it.
Their abilities to lock their limbs may garner interest in the fields regarding camera stabilization due to their similarities in behavior.
In such cases, physics-based models of pigeons that can reproduce such behavior would translate well to engineering.

Preliminary research have proposed possible functionalities behind such behavior, mainly consisting of gathering information of surrounding objects using the retina. However, to our knowledge, no physics-based model capable of replicating the behavior have been proposed. We attempt to make contribution in this aspect in our research.

Our goal in this research is to test such preliminary hypotheses in a physics engine and evaluate their sufficiency in reproducing the head-bobbing behavior.
We conduct such experiment by modeling control systems regarding head-bobbing in pigeons and their morphology, specifically their upper torso, neck, and head.
  We use reinforcement learning for modeling the controller of the pigeon model, because it allows us to construct control systems that represent preliminary hypotheses.

% chapter overview
The following will be discussed in the succeeding chapters.
\begin{description}
  \item [Chapter \ref{ch:background}] Background knowledge of reinforcement learning necessary for understanding our research
  \item [Chapter \ref{ch:preliminary_research}] Preliminary research regarding the use of robotics for testing and proposing hypotheses in biology and the functionalities of head-bobbing in pigeons
  \item [Chapter \ref{ch:approach}] Recapitulation of our goals in this research and the methods that we use for conducting it
  \item [Chapter \ref{ch:experiments}] Details regarding the experiments that we conducted
  \item [Chapter \ref{ch:results}] Results of the aforementioned experiments
  \item [Chapter \ref{ch:discussion}] Analysis of the results and proposals for improvement of our model in future research
  \item [Chapter \ref{ch:conclusion}] Summary of the thesis and conclusion to our research
\end{description}

\chapter{Preliminary Research} \label{ch:preliminary_research}
\section{Modeling Biological Phenomena using Robotics}
% Robotics and Animals paper
  Webb discusses the topic of utilizing robotics to aid research in biology \cite{webb2000does}, where he argues that robotic modeling can be used for proposing or testing hypotheses. This can be accomplished by depicting the hypotheses in the form of algorithms or hardware configurations in robots and observing their emergent behaviors. The generated behaviors can then be compared with their biological counterparts. In particular, robotic models implemented based on preliminary hypotheses can be used as null hypotheses, which can be validated after observing behaviors exhibited by them.

  % Discusses limitations of robotics as biological modeling
    While Webb argues about the potential of using robotics for research in biology, he also highlights the limitations of such methods.
    Since robotic models naturally contain information on all of the assumptions made on the hypotheses of the biological phenomena they reference, discrepancies between the model and the phenomena must be carefully taken into consideration during experiments. For example, if a bipedal robot were to be used to model a pigeon's forward locomotion, it should account for noises and disturbances in the pigeon's sensory input signals and neurological output signals that stimulate their muscles. Additionally, environmental factors such as rough terrain that pigeons walk on should be simulated in the experiment.

  % Algorithmic model vs Robotic model
    From such perspective algorithmic models are inferior to models that utilize real robots since, as Webb states, they cannot confront problems seen in physical environments, such as noises or disturbances, that are not seen in virtual environments, such as physics engines.
    % Where can we use algorithmic models
      As such, algorithms in biology are often used for modeling neurological functionalities. In the context of modeling pigeons' locomotion, algorithmic models can represent the neurological controllers for the pigeons. In this case, the pigeons' interaction between their brains, their morphology, and their surrounding environments can be seen as a control system.
%
%  % Learning model uses
%    % Learning models can be used as an algorithmic modeling method.
%    % Examples
%    % Tie into current research

  % Incremental models
    Webb identifies incremental modeling as a method to tackle the discrepancies between complexity of biological phenomena and their model counterparts. Incremental modeling builds multiple models for the same hypotheses, where each model gets more complex with biological or physical details.

    % Tie into current research
      A biological model of a pigeon with high biological and physical details would consist of the musculoskeletal mechanism of pigeons, a spiking neural network to control all of the muscles, and a surrounding environment with complex details, such as bumpy terrain and colors. This would be too complicated of a problem to tackle in a single research paper, especially considering that the learning and control mechanism of spiking neural networks is currently unclear. Even if the musculoskeletal model were to be controlled by a fully connected deep neural network, the biological and environmental details, including the colors and the sheer number of controllable muscles, would heighten the input and output dimension of the neural network, making training it using deep reinforcement learning extremely difficult.

      Our research specifically focuses on a simplified model of the pigeon, which includes simplified representations of limb control and retinal input.

\section{Head-Bobbing in Pigeons}
% Head-bobbing = Head stabilization + Thrust
  The "head-bob" behavior in pigeons consists of stabilizing the global location and orientation of the head and altering them periodically. Such are dubbed as the hold phase and the thrust phase, respectively \cite{davies1988head}.

% Introduce 2 Hypothesis
  Frost and Davies' have proposed hypotheses regarding the functionalities of such behavior have been proposed \cite{frost1978optokinetic} \cite{davies1988head}. Both proposals highlight the use of the hold phase as a means to stabilize vision and the use of the thrust phase as a means to detect motion parallax and determine the distance between objects.

\subsection{Hold Phase}
% Head stabilization used for stabilizing vision
  Frost's hypothesis links the functionality of the hold phase to the detection of backward motion within the eye \cite{frost1978optokinetic}. Pigeons' heads, while flying or moving forward, would be detecting objects whose movements can be distinguished from the surrounding stationary objects. Since stationary objects would be moving backwards relative to the pigeons' eyes, desensitizing backward motion would be necessary for such distinctions to be detected. However, this desensitization would be detrimental to the pigeons' object recognition while the pigeons' heads are stationary. The hypothesis highlights the existence of "backward notch" cells which counteract the aforementioned desensitization. Such cells would be activated when the pigeons' vision is stabilized and allowing them to distinguish objects moving backward relative to stationary objects, hence the necessity of the hold phase during locomotion.

  Davies' hypothesis challenges this notion and highlights the lack of necessary conditions in Frost's hypothesis to induce a hold phase by stating that "they would fail to detect objects moving backwards through the visual field at velocities similar to that of the bird, as their responses could not be discriminated from those caused by self-induced motion" \cite{davies1988head}. Davies proposes the existence of cells that detect objects' movement relative to stationary backgrounds regardless of their directions.

  In the context of our model, combining the two hypotheses leads to a mechanism that stabilizes the head of the pigeon relative to arbitrary stationary objects and activate cells that detect arbitrary motion during the hold phase.

\subsection{Thrust Phase}
% Motion Parallax
  % Discussed by Frost
    Frost proposes a hypothesis which links the behavior of thrusting the head forward to depth perception \cite{frost1978optokinetic}. He takes the two most common methods of depth perception in biology, stereopsis and motion parallax, into consideration, and points out that, upon examining the anatomy of birds' heads, the eyes are seen to be placed on the opposite sides of the skull instead of having both of them be placed on the frontal area. Such positioning of the eyes in pigeons implies that the overlapping areas in visions of both eyes are reduced compared to animals that mainly use stereopsis for identifying depth of objects. Given such observation, he argues that pigeons are more likely to resort to utilizing motion parallax over the alternative, given the lateral eye placement in pigeons' heads.

  % Davies' contribution
    Davies later elaborates on the hypothesis by modeling the retinal information of objects \cite{davies1988head}.

    Davies models the depth perception caused by motion parallax as
    \begin{equation}
      \dot \theta = \frac {v \sin \theta} {x}
    \end{equation}
    where $\theta$ is the angle of the object relative to the eye, $v$ is the velocity of the pigeon's head, and $x$ is the distance between the object and the head, or the depth of the object. The equation indicates that the depth of the object can be derived if the velocity of the head and the angular velocity of the object within the retina are given.

    Davies further extends this model to account for detecting the difference in depth between objects. Davies differentiates the angular velocity of a single object and the relative angular velocity between an object at depth $x$ and an object at depth $y$ with the notation $\Delta$.
    \begin{equation} \label{equ:davies_motion_parallax}
      \Delta \dot \theta = v \sin \theta \left( {\frac 1 x - \frac 1 y} \right)
    \end{equation}

    Given the two equations, Davies argues that maximizing the speed of head would result in better differentiation between objects with different depths. By thrusting the head forward, the pigeon would be able to detect stationary objects in different depths, which cannot be detected during the hold phase. As an example Davies mentions the role of the eye in detection of food, such as the detection of bread crumbs lying on the ground close by as opposed to lampposts seen far away.

\subsection{Regarding Kinematic Functionalities}
  When building our pigeon model, in addition to the hypotheses proposed regarding the hold phase and the thrust phase, we must take the effect of the torques of the neck joints and the movement of the head generated by them. Intuitively such motion would alter the balance of the entire pigeon, leading to mutual adjustments between the bipedal walk cycle and the neck control for head positioning. Additionally, the head-bobbing motion could be hypothesized to function as a means to balance the pigeon's forward locomotion. However as Davies argues in his paper \cite{davies1988head}, since head-bobbing is not exhibited during fast forward locomotion, such as flying, it is unlikely that such behavior has kinematic purposes. Frost's findings \cite{frost1978optokinetic} further support this idea by demonstrating that pigeons stabilize their head in one global coordinate regardless of the body's global velocity, it is likely that head-bobbing's functionalities are solely based on vision.

\section{Preliminary Research}
\subsection{Modeling biological phenomena}
% Robotics and Animals paper
  % Discusses limitations of robotics as biological modeling
    % Robotic modeling can be used for proposing or testing hypothesis
    % Discrepancies between the model and the organism that is being referenced
  % Algorithmic model vs Robotic model
    % Algorithmic models inferior to robotic models
    % Where can we use algorithmic models
      % Neurological functionalities
    % Connect to this pigeon research
  % Learning model uses
    % Examples
    % Tie into current research
  % Incremental models
    % Method to tackle discrepancies between complexity of biological phenomena and its model
    % Tie into current research

\subsection{Head-Bobbing in Pigeons}
% Head-bobbing = Head stabilization
  % The "head-bob" behavior in pigeons consists of stabilizing the location and orientation of the head and altering them periodically. Such are dubbed as the hold phase and the thrust phase, respectively \cite{frost_1978}.

% Introduce 2 Hypothesis
  % Frost and Davies' have proposed hypothesis regarding the functionalities of such behavior have been proposed \cite{frost_1978} \cite{davies_1988}. Both proposals highlight the use of the hold phase as a means to stabilize vision and the use of the thrust phase as a means to detect motion parallax and determine the distance between objects. 
% Head stabilization used for stabilizing vision
% Motion Parallax

% Kinematic purpose

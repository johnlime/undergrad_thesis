\section{Preliminary Research}
\subsection{Modeling biological phenomena}
% Robotics and Animals paper
  % Discusses pros and limitations of robotics as biological modeling
  % Algorithmic model vs Robotic model
    % Algorithmic models inferior to robotic models
    % Where can we use algorithmic models
    % Connect to this pigeon research
  % Learning model uses
    % Examples
    % Tie into current research
  % Incremental models
    % Method to tackle discrepancies between complexity of biological phenomena and its model
    % Tie into current research

\subsection{Head-Bobbing in Pigeons}
% Head-bobbing = Head stabilization
  % The "head-bob" behavior in pigeons consists of stabilizing the location and orientation of the head and altering them periodically. Such are dubbed as the hold phase and the thrust phase, respectively \cite{frost_1978}.
% Head stabilization used for stabilizing vision
  % Mainly 2 hypothesis regarding the functionalities of such behavior have been proposed \cite{frost_1978} \cite{davies_1988}.
  Both hypothesis suggest that the hold phase is utilized for stabilizing vision.
% Introduce 2 Hypothesis
% How each of the 2 hypothesis diverge
  % What this explains
  % What the shortcomings are
% Side note: 2 more hypothesis mentioned in the two papers that we are probably going to ignore

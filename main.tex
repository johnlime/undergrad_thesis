%%This is the KMD un-official thesis template (Engineering Format for English Writer).

\documentclass[12pt]{report}
\usepackage{kmd-emthesis}
\usepackage{graphicx}
\usepackage{subfigure}
\usepackage{tabularx}
\usepackage{longtable}
\usepackage{multirow}
\usepackage{url}
\usepackage{fancybox}
\usepackage{amssymb}
\usepackage{moreverb}
\usepackage{afterpage}
\usepackage{cite}
%\usepackage{harvard-oikos}
\usepackage{ccaption} % caption
\usepackage{fn2end} %footnotes script
\usepackage{fn2end_config} %footnotes edit script
\usepackage{indentfirst}
\usepackage{float}
\usepackage{listings}
\usepackage{amsmath}
\lstset{
  basicstyle=\small\ttfamily,
  columns=flexible,
  breaklines=true
}
%%%%%%%%%%%%%%%%%%%%%%%%%%%%

\makeatother

% Page Style Settings
\pagestyle{final}       % Final Draft
%\pagestyle{draft}      % Drafts

% Language Settings
\lang{English} % English

% Student ID number
\studentnumber{71744696}

% Choosing Masters Thesis or Report
\doctitle{\bachelorsreport}

% What Degree to obtain
\major{\envandinfo}

% Title (in LaTeX)
\etitle{Modeling Head-Bobbing in Pigeon Locomotion using Reinforcement Learning}

% Title (in plain text)
%   No need to set if the same as  (in LaTeX)
\eptitle{Practice of Design Thinking Workshop to Develop ``Media Innovator'' Leading Creative Society}


%Author's Name (in LaTeX)
\eauthor{Mioto Takahashi}

% Author's Name (in plain text)
%   No need to set if the same as in (in LaTeX)
% \epauthor{Mioto Takahashi}


% Academic Year
\syear{2022}
%\heiseiyear{24}
%\smonth{2}
%\sday{7}


% Advisors
\ecomembers{Professor Tatsuya Hagino}{(Supervisor)}
           {Professor Takashi Hattori}{(Co-Supervisor)}
           {}{}
           {}{}
\eaddcomembers{}{}

% 5 or 6 Keywords (in LaTeX)
\ekeywords{Reinforcement Learning, Biomimetic, Pigeon, Locomotion}

% 5 or 6 Keywords (in plain text)
% \epkeywords{Design Thinking, Creative Society, Workshop, Innovation, Education}

%Choose one submission category from [Design, Science / Engineering, Social Science / Humanities, Action Research]
\ecategory{Science / Engineering}
%Choose one submission category from [Design, Science / Engineering, Social Science / Humanities, Action Research] (in plain text)
% \ecategory{Science / Engineering}


%%%%%%%%%%%%%%%%%%%%%%%%%%Abstract%%%%%%%%%%%%%%%%%%%%%%%%%%%%%%%%%
\eabstract{

Lorem ipsum dolor sit amet, consectetur adipiscing elit. In efficitur porta augue, at interdum nunc lobortis at. Morbi feugiat facilisis justo, vitae maximus dolor. Cras convallis at elit in porta. Fusce lobortis tortor nibh, quis imperdiet arcu luctus quis. Mauris imperdiet urna eu mauris aliquet, vitae tincidunt orci dapibus. Vestibulum convallis elit ut velit accumsan cursus. Pellentesque lacus lacus, blandit eu felis vitae, pellentesque dignissim est.
}
%%%%%%%%%%%%%%%%%%%%%%%%% Document starts here %%%%%%%%%%%%%%%%%%%%%%%%%%%%


\begin{document}
  \titlepage
  \comemberspage
  \firstabstract
  %\secondabstract

  % Table of Contents
  \toc
  \newpage
  \listoffigures
  %\listoftables


  \newpage
  \pagenumbering{arabic}
  % Chapters
  \chapter{Background}
\section{Reinforcement Learning}
% rl feedback loop and reward system
  Reinforcement learning is a type of control system that attempts to execute tasks described by manually set cost functions by minimizing them using optimization algorithms or learning algorithms. In the case of deep reinforcement learning, the controller is modeled using deep neural networks and the cost function is minimized using gradient descent algorithms.

  Reinforcement learning divides control systems into an agent and an environment. The agent acts the controller of the system that inspects its environment's state and sends output signals, or actions, that affect the environment. This mutual interactions creates a feedback loop between the two modules.

  In the context of a pigeon tasked to move forward, the pigeon's brain and its nervous system connected to each limb act as the agent, and its surroundings, such as the ground and arbitrary objects on it, act as the environment. The environment outputs a state, such as the global position of the pigeon, which is used as the input for the agent. Using the provided state, the agent calculates and outputs an action, such as the torque of each joint in the pigeon's body. The action alters the state of the environment, and the environment outputs a new state and a reward. The reward describes how well the pigeon was able to execute its task, such as the current position of the pigeon relative to its previous timestep. The agent is updated to output sequences of actions that maximizes the cumulative reward, or return. The return can be interpreted as the inverted or negated cost function.

% using reward as a cost function / constraint / task def
% SAC

  \section{Preliminary Research}
\subsection{Modeling biological phenomena}
% Robotics and Animals paper
  % Discusses pros and limitations of robotics as biological modeling
  % Algorithmic model vs Robotic model
    % Algorithmic models inferior to robotic models
    % Where can we use algorithmic models
    % Connect to this pigeon research
  % Learning model uses
    % Examples
    % Tie into current research
  % Incremental models
    % Method to tackle discrepancies between complexity of biological phenomena and its model
    % Tie into current research

\subsection{Head-Bobbing in Pigeons}
% Head-bobbing = Head stabilization
  % Despite the appearance of a horizontal oscillation, the time lapse of the "head-bob" behavior in pigeons consists of stabilizing the location and orientation of the head and altering them when they become unreachable. Such are dubbed as the hold phase and the thrust phase \cite{frost_1978}.
% Head stabilization used for stabilizing vision
  % Several hypothesis regarding the functionalities of such behavior have been proposed, all of which that we focus in this research suggest that the hold phase is utilized for stabilizing vision.
% Introduce 4 Hypothesis
% How each of the 4 hypothesis diverge
  % What this explains
  % What the shortcomings are

  \chapter{Approach}
% def of pigeon model
\section{Definition of the Pigeon Model}
  We define a simplified 2-dimensional model of pigeons based on incremental modeling. The pigeon model consists of 3 joints connecting one body representing the head, 2 bodies representing the neck, and one body representing the torso. The model's physics, mainly the collision and gravity, is simulated in a 2 dimensional physics engine. The torso's orientation and y-position is fixed while its x-position is incremented by a constant value. This represents forward locomotion at a constant speed.

  Additionally, we build control systems for the model using deep reinforcement learning. By using deep reinforcement learning, we can train the controller to maximize reward functions that represent hypotheses or manually-defined trajectories for the bodies in the model to follow.

\section{Baseline: Manually-Defined Head Trajectory}
  As the baseline for the model's control system, we attempt to recreate the head-bob movement by setting a target position for the head's position to match every timestep. The target position is first defined at a set location in front of the pigeon model $T$ relative to the position of its torso. The target then acts as a static position in the global coordinate for the head to follow. If the distance between the target position and the torso's position goes below a set threshold value, the target is repositioned at the same location $T$ relative to the torso's position.

\section{Hypothesis Testing}
  We compare the trajectories of the bodies in the baseline control system to those generated by the control system that represents preliminary hypotheses.

  %% description below should assume that we already know the angle of objects within retina
  Preliminary hypotheses for the functionalities of head-bobbing behavior can be depicted using two reward functions, each representing head stabilization and motion parallax generation.

  % Hold phase: Head stabilization
  \subsection{Head Stabilization}
    Davies' hypothesis indicate that static objects should be stabilized into one location in the retina for the pigeon to easily determine the moving objects' velocities during the hold phase. In application, the pigeon's head should move in a trajectory that minimizes retinal velocities of objects.

    We define the reward function for head stabilization as,
    \begin{equation}
      r_t = - \sum_i^n |\dot \theta_i|
    \end{equation}
    where $n$ is the number of objects in the environment and $\dot \theta$ is the angular velocity of each object.

  % Thrust phase: Maximizing motion parallax
  \subsection{Motion Parallax}
    %
    % sum of all velocities of objects should be maximized


  % can hold and thrust phases emerge if given such tradeoff?
  % how would optimization algorithms solve this tradeoff?

  \chapter{Experiments}

% dimensions of the pigeon model
  % Our pigeon model's dimensions and orientations are set at static values for all experiments, as shown in [Figure ???].
    % insert diagram
  % Additionally, the pigeon's head relative to the body is facing the negative direction relative to the x-axis.
  % The widths and heights of each limb, head, and body are $(10, 4)$, $(6, 4)$, $(20, 10)$ respectively.
  % The initial angles of each limb are oriented at 45 degrees relative to the x-axis, and both the head and the body are oriented parallel to the x axis.
  % The body's initial position is at the origin, and is set to move at a constant speed in the negative direction along the x-axis.

% head trajectory setting
  % The target head location $T$ is set at $(0, -2)$ relative to the initial position of the head.
  % The threshold value for the distance between the target position and the torso's position is set at 10.
  % We set a value that represents the margin of error between $T$ and the position of the head, which allows positive reward to be returned from the environment if the position of the head is within it.

% external objects
  % 3 points and their positions are defined to represent 1 static and 2 dynamic objects placed on the surrounding environment of the pigeon.
  % The static object's position is $(-30.0, 30.0)$, and the 2 dynamic objects' positions are $(-30.0, 60.0)$, $(-60.0, 30.0)$. The former dynamic object moves at speed 1 in the positive direction along the x-axis, while the latter moves at the same speed in the negative direction along the x-axis.

% experiments
  % _head_stable_manual_reposition
  % _head_stable_manual_reposition_strict_angle
  % _retinal_stabilization
  % _motion_parallax
  % _fifty_fifty

% We constructed OpenAI Gym environments $PigeonEnv3Joints$ and $PigeonRetinalEnv$ for conducting reinforcement learning based on the baseline training and preliminary hypotheses, respectively.
  % Details regarding the environments' code can be referred to the Appendix.

% SAC a type of reinforcement learning model
  % epochs
  % timesteps
% why we didn't use PPO
  % Baseline for many rl experiments
    % https://scholar.google.com/scholar?as_ylo=2018&q=proximal+policy+optimization&hl=en&as_sdt=0,5
    % https://arxiv.org/abs/1905.01360 (pommerman)
  % we tested it, and SAC seemed to have a more stable learning curve; thought that it would be more reliable
  % SAC represents the exploration of new trajectories of actions more; exploration of better execution seen in biological organisms' behaviors

  \chapter{Results}
% summary of the behavioral results per experiment
  We summarize the behaviors exhibited by the pigeon models in each experiment, categorized by their body speeds and the reward functions the controllers were trained with.

  \begin{itemize}
    \item Body speed 0; $r_{head\_stable\_manual\_reposition\_strict\_angle}$
    \begin{description}
      \item The pigeon model managed to keep its head stationary as expected.
    \end{description}

    \item Body speed 1; $r_{head\_stable\_manual\_reposition\_strict\_angle}$
    \begin{description}
      \item The pigeon model initially struggled to control its head in a specific pattern or position.
      However, after 5 seconds, it managed to align the head within the radius $max\_offset$ around the manually set position $T$, resulting in a generation of trajectory that depicts a pattern of thrust and hold phases.
      The final 5 seconds of the footage depicts the head and the limbs getting stuck on the topside of the body.
    \end{description}

    \item Body speed 1; $r_{head\_stable\_manual\_reposition}$
    \begin{description}
      \item The pigeon model produced similar results to the previous experiment, with the exception of the overall performance.
      Due to the less strict definition of the reward function, it generated a trajectory that resulted in the acquirement of larger return, and as a result, a better performance in producing the desired head-bobbing behavior.
      In particular, the timespan of the head following the manually-defined trajectory was longer than those seen in the previous experiment.
    \end{description}

    \item Body speed 0; $r_{fifty\_fifty}$
    \begin{description}
      \item The head of the model moved around in the vicinity of a single position, similar to the baseline counterpart.
    \end{description}

    \item Body speed 1; $r_{fifty\_fifty}$
    \begin{description}
      \item The head of the model gradually leaned down and backwards as if it were following the surrounding stationary object.
    \end{description}

  \end{itemize}

% raw trajectories will NOT be included!
% figures for behavior renderings explanation
  We rendered the resulting behaviors produced by controllers trained on aforementioned reward functions and environments into images or frames.
  Combining the frames generated for each of the 1000 timesteps and setting as 60 frames per second resulted in 33.35 second videos.
  The time-lapses presented in Figures \ref{fig:manual_trajectory_body_speed_0}, \ref{fig:manual_trajectory_strict_body_speed_1}, \ref{fig:manual_trajectory_not_strict_body_speed_1}, \ref{fig:fifty_fifty_body_speed_0}, and \ref{fig:fifty_fifty_body_speed_1} were created by sampling every $300$ frames within the last 30 seconds of the video.
  The frames' sequential order is from the top left to the bottom right.
  The camera is locked to follow the pigeon's body.
  % 60/5 * [10] = 300

% figures of pigeon behavior renderings
  \begin{figure}[H]
      \centering
      \includegraphics[width=1\textwidth]{figures/frames/frames_001.png}
      \caption{Control of a pigeon model with a static body trained on $r_{head\_stable\_manual\_reposition\_strict\_angle}$ with $max\_offset = 0.5$. The green circle indicate the margin of error around the target head location defined by $max\_offset$.}
      \label{fig:manual_trajectory_body_speed_0}
  \end{figure}

  \begin{figure}[H]
      \centering
      \includegraphics[width=1\textwidth]{figures/frames/frames_002.png}
      \caption{Control of a pigeon model with the body speed of 1 trained on $r_{head\_stable\_manual\_reposition\_strict\_angle}$ with $max\_offset = 1.0$. The green circle indicate the margin of error around the target head location defined by $max\_offset$.}
      \label{fig:manual_trajectory_strict_body_speed_1}
  \end{figure}

  \begin{figure}[H]
      \centering
      \includegraphics[width=1\textwidth]{figures/frames/frames_003.png}
      \caption{Control of a pigeon model with the body speed of 1 trained on $r_{head\_stable\_manual\_reposition}$ with $max\_offset = 1.0$. The green circle indicate the margin of error around the target head location defined by $max\_offset$.}
      \label{fig:manual_trajectory_not_strict_body_speed_1}
  \end{figure}

  \begin{figure}[H]
      \centering
      \includegraphics[width=1\textwidth]{figures/frames/frames_004.png}
      \caption{Control of a pigeon model with a static body trained on $r_{fifty\_fifty}$}
      \label{fig:fifty_fifty_body_speed_0}
  \end{figure}

  \begin{figure}[H]
      \centering
      \includegraphics[width=1\textwidth]{figures/frames/frames_005.png}
      \caption{Control of a pigeon model with the body speed of 1 trained on $r_{fifty\_fifty}$}
      \label{fig:fifty_fifty_body_speed_1}
  \end{figure}

% figures of pigeon head trajectories

  \chapter{Analysis}
% may need a state machine-like mechanism for replication of head-bob
% hierarchical reinforcement learning; hierarchical control system

  %\chapter{Discussion}
% The results indicate that
  % regarding the behavior exhibited by the pigeon model with body speed 1 whose deep reinforcement learning controller was trained on $r_{fifty\_fifty}$
  % while the combination of visual stabilization and motion parallax are sufficient to generate a behavior where the pigeon's head is fixed in one position
  % insufficient to produce head-bobbing behaviors in moving bodies during forward locomotion

  % did not need to constantly thrust head forward to maximize Davies' equation depicting motion parallax between objects
    % visual stabilization with retinal cells capable of detecting movement in all directions is enough to maximize the sum of external objects' angular velocities within the retina
    % Davies' equation only works when all objects within the retina are globally static and does not account for moving objects.
    % Since the external objects in our experiment had objects moving forwards and backwards, such may have automatically increased the reward function depicting Davies' equation of motion parallax per timestep.


% may need a state machine-like mechanism for replication of head-bob
  % composed of the "hold phase" state and the "thrust phase" state
% hierarchical control system
  % hierarchical reinforcement learning should be used for modeling the control system with this hypothesis
% indicates that a hierarchical control system is embedded in pigeons' neurology.
% pattern generating module or functionality, such as central pattern generators seen in the spinal cortex

% head goes downwards during forward locomotion
% lack of muscular strain penalty?

% needs higher details, such as an addition of muscular physics, and progression in incremental modeling
% muscular simulation and their placements upon the skeletal model may lead to more stabilization (cite Geijtenbeek)

  %\input{futurework.tex}


  % Acknowledgements
  \acknowledgements
  Lorem ipsum dolor sit amet, consectetur adipiscing elit. In efficitur porta augue, at interdum nunc lobortis at. Morbi feugiat facilisis justo, vitae maximus dolor. Cras convallis at elit in porta. Fusce lobortis tortor nibh, quis imperdiet arcu luctus quis. Mauris imperdiet urna eu mauris aliquet, vitae tincidunt orci dapibus. Vestibulum convallis elit ut velit accumsan cursus. Pellentesque lacus lacus, blandit eu felis vitae, pellentesque dignissim est.



  % Reference
  \newpage
  %\reference
  \nocite{*}
  %\nocite{*} %Use if you want to list everything listed in bibtex, if not comment it out

  %Bibliography
  %\bibliographystyle{abbrv}

  %ACM SIGCHI Style
  \bibliographystyle{acm-sigchi}
  \bibliography{ref}

  % Appendix
  \appendix
  \section{OpenAI Gym for Simplified Model of Pigeons' Head Control}
% written in Python

% reference to OpenAI Gym library site or paper

\subsection{Usage}
\begin{lstlisting}
PigeonEnv3Joints(self, body_speed = 0,
                 reward_code = "head_stable_manual_reposition",
                 max_offset = 0.5)
\end{lstlisting}
\begin{itemize}
    \item \lstinline|body_speed| indicates the speed in which the pigeon model's body moves.

    \item Each of the following \lstinline|reward_code| are assigned to their respective reward functions.
        \begin{itemize}
          \item \lstinline|"head_stable_manual_reposition"|
              \begin{description}
                  $r_{head\_stable\_manual\_reposition}$
              \end{description}
          \item \lstinline|"head_stable_manual_reposition_strict_angle"|
              \begin{description}
                  $r_{head\_stable\_manual\_reposition\_strict\_angle}$
              \end{description}
        \end{itemize}
    \item \lstinline|max_offset| indicates the $max\_offset$ for each reward function to reference.

\begin{lstlisting}
PigeonRetinalEnv(self,
                 body_speed = 0,
                 reward_code = "motion_parallax")
\end{lstlisting}

Parallel to \lstinline|PigeonEnv3Joints|, each of the following \lstinline|reward_code| are assigned to their respective reward functions.
\begin{itemize}
  \item \lstinline|"retinal_stabilization"|
      \begin{description}
          Depicts the preliminary hypothesis regarding the functionality of retinal stabilization during the hold phase.
      \end{description}
  \item \lstinline|"motion_parallax"|
      \begin{description}
          Depicts the preliminary hypothesis regarding the functionality of motion parallax induced depth perception during the thrust phase.
      \end{description}
  \item \lstinline|"fifty_fifty"|
      \begin{itemize}
          \item $r_{head\_stable\_manual\_reposition\_strict\_angle}$
          \item Sum of rewards produced by
      \end{itemize}
\end{itemize}

% dependencies
\subsection{Dependencies (Anaconda YAML File)}
The listed versions are recommendations and not strictly necessary for replication
\begin{lstlisting}
name: pigeon-env
channels:
  - conda-forge
  - defaults
dependencies:
  - bzip2=1.0.8=h0d85af4_4
  - ca-certificates=2021.10.8=h033912b_0
  - certifi=2016.9.26=py36_0
  - ffmpeg=4.3.2=h4dad6da_0
  - freetype=2.10.4=h4cff582_1
  - future=0.18.2=py36h79c6626_3
  - gettext=0.19.8.1=h7937167_1005
  - gmp=6.2.1=h2e338ed_0
  - gnutls=3.6.13=h756fd2b_1
  - lame=3.100=h35c211d_1001
  - libcxx=12.0.0=h2f01273_0
  - libffi=3.3=hb1e8313_2
  - libiconv=1.16=haf1e3a3_0
  - libpng=1.6.37=h7cec526_2
  - ncurses=6.3=hca72f7f_2
  - nettle=3.6=hedd7734_0
  - openh264=2.1.1=hfd3ada9_0
  - openssl=1.1.1l=h0d85af4_0
  - pip=21.2.2=py36hecd8cb5_0
  - pybox2d=2.3.10=py36hefe7e0e_1
  - pyglet=1.5.16=py36h79c6626_0
  - python=3.6.13=h88f2d9e_0
  - python_abi=3.6=2_cp36m
  - readline=8.1.2=hca72f7f_1
  - setuptools=58.0.4=py36hecd8cb5_0
  - sqlite=3.37.0=h707629a_0
  - tk=8.6.11=h7bc2e8c_0
  - wheel=0.37.1=pyhd3eb1b0_0
  - x264=1!161.3030=h0d85af4_1
  - xz=5.2.5=h1de35cc_0
  - zlib=1.2.11=h4dc903c_4
  - pip:
    - cloudpickle==2.0.0
    - gym==0.21.0
    - importlib-metadata==4.8.3
    - numpy==1.19.5
    - typing-extensions==4.0.1
    - zipp==3.6.0
\end{lstlisting}

\subsection{Manually Defined Head Trajectory (Baseline)}
% Python code as of 01-08-22
\begin{lstlisting}
from Box2D import *
import gym
from gym import spaces

from math import sin, pi, sqrt
import numpy as np
from copy import copy, deepcopy

# anatomical variables ("macros")
BODY_WIDTH = 10
BODY_HEIGHT = 5

LIMB_WIDTH = 5
LIMB_HEIGHT = 2

HEAD_WIDTH = 3

ANGLE_FREEDOM = 0.6

# control variables/macros
MAX_JOINT_TORQUE = 200 #70
MAX_JOINT_SPEED = 5 #10
VELOCITY_WEIGHT = 1.0 #0.9
LIMB_DENSITY = 0.1 ** 3
LIMB_FRICTION = 5

VIEWPORT_SCALE = 6.0
FPS = 60

HEAD_OFFSET_X = 10
HEAD_OFFSET_Y = 2

class PigeonEnv3Joints(gym.Env):
    metadata = {"render.modes": ["human", "rgb_array"], "video.frames_per_second": FPS}

    def __init__(self,
                 body_speed = 0,
                 reward_code = "head_stable_manual_reposition",
                 max_offset = 0.5):
        """
        Action and Observation space
        """

        # 3-dim joints' torque ratios
        self.action_space = spaces.Box(
            np.array([-1.0] * 3).astype(np.float32),
            np.array([1.0] * 3).astype(np.float32),
        )
        # 2-dim head location;
        # 1-dim head angle;
        # 3x2-dim joint angle and angular velocity;
        # 1-dim x-axis of the body
        # [NEW] 2-dim target head location
        high = np.array([np.inf] * 12).astype(np.float32) # formally 10
        self.observation_space = spaces.Box(-high, high)

        """
        Box2D Pigeon Model Params and Initialization
        """
        self.world = b2World()                          # remove in Framework
        self.body = None
        self.joints = []
        self.head = None
        self.bodyRef = [] # for destruction
        self.body_speed = body_speed
        self._pigeon_model()

        """
        Box2D Simulation Params
        """
        self.timeStep = 1.0 / FPS
        self.vel_iters, self.pos_iters = 10, 10

        self.viewer = None

        """
        Assigning a Reward Function
        """
        self._assign_reward_func(reward_code, max_offset)

    """
    Define Reward Function and Necessary Parameters
    """
    def _assign_reward_func(self, reward_code, max_offset):
        if "head_stable_manual_reposition" in reward_code:
            self.max_offset = max_offset

            self.relative_repositioned_head_target_location = np.array(self.head.position) - np.array([0, HEAD_OFFSET_Y])
            self.head_target_location = self.relative_repositioned_head_target_location + np.array(self.body.position)
            self.head_target_angle = self.head.angle
            self.reward_function = self._head_stable_manual_reposition

            if "strict_angle" in reward_code:
                self.reward_function = self._head_stable_manual_reposition_strict_angle

        else:
            raise ValueError("Unknown reward_code")

    """
    Box2D Pigeon Model
    """
    def _pigeon_model(self):
        # params
        body_anchor = np.array([float(-BODY_WIDTH), float(BODY_HEIGHT)])
        limb_width_cos = LIMB_WIDTH / sqrt(2)

        self.bodyRef = []
        # body definition
        self.body = self.world.CreateKinematicBody(
            position = (0, 0),
            shapes = b2PolygonShape(box = (BODY_WIDTH, BODY_HEIGHT)), # x2 in direct shapes def
            linearVelocity = (-self.body_speed, 0),
            angularVelocity = 0,
            )
        self.bodyRef.append(self.body)

        # neck as limbs + joints definition
        self.joints = []
        current_center = deepcopy(body_anchor)
        current_anchor = deepcopy(body_anchor)
        offset = np.array([-limb_width_cos, limb_width_cos])
        prev_limb_ref = self.body
        for i in range(2):
            if i == 0:
                current_center += offset

            else:
                current_center += offset * 2
                current_anchor += offset * 2

            tmp_limb = self.world.CreateDynamicBody(
                position = (current_center[0], current_center[1]),
                fixtures = b2FixtureDef(density = LIMB_DENSITY,
                                        friction = LIMB_FRICTION,
                                        restitution = 0.0,
                                        shape = b2PolygonShape(
                                            box = (LIMB_WIDTH, LIMB_HEIGHT)),
                                        ),
                angle = -pi / 4
            )
            self.bodyRef.append(tmp_limb)

            tmp_joint = self.world.CreateRevoluteJoint(
                bodyA = prev_limb_ref,
                bodyB = tmp_limb,
                anchor = current_anchor,
                lowerAngle = -ANGLE_FREEDOM * b2_pi, # -90 degrees
                upperAngle = ANGLE_FREEDOM * b2_pi,  #  90 degrees
                enableLimit = True,
                maxMotorTorque = MAX_JOINT_TORQUE,
                motorSpeed = 0.0,
                enableMotor = True,
            )

            self.joints.append(tmp_joint)
            prev_limb_ref = tmp_limb

        # head def + joints
        current_center += offset
        current_anchor += offset * 2
        self.head = self.world.CreateDynamicBody(
            position = (current_center[0] - HEAD_WIDTH, current_center[1]),
            fixtures = b2FixtureDef(density = LIMB_DENSITY,
                                    friction = LIMB_FRICTION,
                                    restitution = 0.0,
                                    shape = b2PolygonShape(
                                        box = (HEAD_WIDTH, LIMB_HEIGHT)),
                                    ),
        )
        self.bodyRef.append(self.head)

        head_joint = self.world.CreateRevoluteJoint(
            bodyA = prev_limb_ref,
            bodyB = self.head,
            anchor = current_anchor,
            lowerAngle = -ANGLE_FREEDOM * b2_pi, # -90 degrees
            upperAngle = ANGLE_FREEDOM * b2_pi,  #  90 degrees
            enableLimit = True,
            maxMotorTorque = MAX_JOINT_TORQUE,
            motorSpeed = 0.0,
            enableMotor = True,
        )
        self.joints.append(head_joint)

        # head tracking
        self.head_prev_pos = np.array(self.head.position)
        self.head_prev_ang = self.head.angle

    def _destroy(self):
        for body in self.bodyRef:
            # all associated joints are destroyed implicitly
            self.world.DestroyBody(body)

    def _get_obs(self):
        # (self.head{relative}, self.joints -> obs) operation
        obs = np.array(self.head.position) - np.array(self.body.position)
        obs = np.concatenate((obs, self.head.angle), axis = None)
        for i in range(len(self.joints)):
            obs = np.concatenate((obs, self.joints[i].angle), axis = None)
            obs = np.concatenate((obs, self.joints[i].speed), axis = None)
        obs = np.concatenate((obs, self.body.position[0]), axis = None)

        # complement a target position
        obs = np.concatenate((obs, self.head_target_location - np.array(self.body.position)),
                              axis = None)

        obs = np.float32(obs)
        assert self.observation_space.contains(obs)
        return obs

    def reset(self):
        self._destroy()
        self._pigeon_model()
        return self._get_obs()

    def _head_target_reposition_mechanism(self):
        # detect whether the target head position is behind the body edge or not
        if self.head_target_location[0] > self.body.position[0] - float(BODY_WIDTH + HEAD_OFFSET_X):
            self.head_target_location = np.array(self.body.position) + \
                self.relative_repositioned_head_target_location

        head_dif_loc = np.linalg.norm(np.array(self.head.position) - \
                self.head_target_location)
        head_dif_ang = abs(self.head.angle - self.head_target_angle)
        return head_dif_loc, head_dif_ang

    """
    Modular Reward Functions
    """
    def _head_stable_manual_reposition(self):
        # This method is separated from step(), since there are variables used
        # that are only defined in with this strain of reward functions
        head_dif_loc, head_dif_ang = self._head_target_reposition_mechanism()

        reward = 0
        # threshold reward function with static offset
        if head_dif_loc < self.max_offset:
            reward += 1 - head_dif_loc/self.max_offset

            if head_dif_ang < np.pi / 6: # 30 deg
                reward += 1 - head_dif_ang/ np.pi

        return reward

    def _head_stable_manual_reposition_strict_angle(self):
        head_dif_loc, head_dif_ang = self._head_target_reposition_mechanism()

        reward = 0
        # threshold reward function with static offset
        if head_dif_loc < self.max_offset:
            if head_dif_ang < np.pi / 6: # 30 deg
                reward += 1 - head_dif_ang/ np.pi

        return reward

    def step(self, action):
        assert self.action_space.contains(action)
        # self.world.Step(self.timeStep, self.vel_iters, self.pos_iters)
        # Framework handles this differently
        # Referenced bipedal_walker
        # self.world.Step(1.0 / 50, 6 * 30, 2 * 30)
        self.world.Step(1.0 / FPS, self.vel_iters, self.pos_iters)
        obs = self._get_obs()

        # MOTOR CONTROL
        for i in range(len(self.joints)):
            # Copied from bipedal_walker
            self.joints[i].motorSpeed = float(MAX_JOINT_SPEED * (VELOCITY_WEIGHT ** i) * np.sign(action[i]))
            self.joints[i].maxMotorTorque = float(
                MAX_JOINT_TORQUE * np.clip(np.abs(action[i]), 0, 1)
            )

        reward = self.reward_function()

        done = False
        info = {}
        return obs, reward, done, info

    def render(self, mode = "human"):
        from gym.envs.classic_control import rendering
        if self.viewer is None:
            self.viewer = rendering.Viewer(500, 500)

            # Set ORIGIN POINT relative to camera
            self.camera_trans = b2Vec2(-250, -200) \
            + VIEWPORT_SCALE * self.bodyRef[0].position # camera moves with body

            ## Needs head_stable_manual_reposition reward function to execute
            try:
                # init visualize max_offset
                render_target_area = rendering.make_circle( \
                    radius=VIEWPORT_SCALE * self.max_offset,
                    res=30,
                    filled=True)
                target_translate = rendering.Transform(
                    translation = VIEWPORT_SCALE * self.head_target_location - self.camera_trans,
                    rotation = 0.0,
                    scale = VIEWPORT_SCALE * np.ones(2)
                )
                render_target_area.add_attr(self.target_translate)
                render_target_area.set_color(0.0, 1.0, 0.0)
                self.viewer.add_geom(render_target_area)
            except:
                pass

            # init translation and rotation for each limb
            self.render_polygon_list = []
            self.render_polygon_rotate_list = []
            self.render_polygon_translate_list = []
            for body in self.bodyRef:
                polygon = rendering.FilledPolygon(
                    body.fixtures[0].shape.vertices
                )
                rotate = rendering.Transform(
                    translation = (0.0, 0.0),
                    rotation = body.angle,
                )
                translate = rendering.Transform(
                    translation = VIEWPORT_SCALE * body.position - self.camera_trans,
                    rotation = 0.0,
                    scale = VIEWPORT_SCALE * np.ones(2)
                )
                polygon.set_color(1.0, 0.0, 0.0)
                polygon.add_attr(rotate)
                polygon.add_attr(translate)
                self.render_polygon_list.append(polygon)
                self.render_polygon_rotate_list.append(rotate)
                self.render_polygon_translate_list.append(translate)
                self.viewer.add_geom(polygon)

        # Update ORIGIN POINT relative to camera
        self.camera_trans = b2Vec2(-250, -200) \
        + VIEWPORT_SCALE * self.bodyRef[0].position # camera moves with body

        ## Needs head_stable_manual_reposition reward function to execute
        try:
            # update max_offset shape translation
            new_target_translate = VIEWPORT_SCALE * self.head_target_location - self.camera_trans
            self.target_translate.set_translation(new_target_translate[0], new_target_translate[1])
        except:
            pass

        # update body rotation and translation
        for i, body in enumerate(self.bodyRef):
            self.render_polygon_rotate_list[i].set_rotation(body.angle)
            new_body_translate = VIEWPORT_SCALE * body.position - self.camera_trans
            self.render_polygon_translate_list[i].set_translation(new_body_translate[0], new_body_translate[1])

        return self.viewer.render(return_rgb_array = mode == "rgb_array")

    def close(self):
        # self._destroy()
        # self.world = None

        if self.viewer:
            self.viewer.close()
            self.viewer = None
\end{lstlisting}

  \subsection{Pigeons' Head Control Based on Retinal Inputs}
% Python code as of 01-15-22
\begin{lstlisting}
import PigeonEnv3Joints, VIEWPORT_SCALE
import numpy as np
import gym
from gym import spaces

class PigeonRetinalEnv(PigeonEnv3Joints):

    def __init__(self,
                 body_speed = 0,
                 reward_code = "motion_parallax"):

        """
        Object Location Init (2D Tensor)
        """
        self.objects_position = np.array([[-30.0, 30.0],
                                          [-30.0, 60.0],
                                          [-60.0, 30.0],])
        self.objects_velocity = np.array([[0.0, 0.0],
                                          [1.0, 0.0],
                                          [-1.0, 0.0],])

        """
        Init based on superclass
        Reward function is defined here
        """
        super().__init__(body_speed, reward_code)

        """
        Redefining Observation space
        """
        # 2-dim head location;
        # 1-dim head angle;
        # 3x2-dim joint angle and angular velocity;
        # 1-dim x-axis of the body
        high = np.array([np.inf] * 10).astype(np.float32) # formally 10
        self.observation_space = spaces.Box(-high, high)


    """
    Retinal coords (angles); Within [-np.pi, np.pi]
    """
    def _get_retinal(self, object_position):
        # normalized direction of object from head
        object_direction = object_position - np.array(self.head.position)
        object_direction = object_direction / np.linalg.norm(object_direction)

        sign = np.ones(object_direction.shape[0])
        for i in range(sign.size):
            # is the object above or below the head?
            if object_direction[i][1] < 0:
                sign[i] = -1

        # calculate COSINE angle of object relative to head (positive if above, negative if below)
        # cosine_angle is of size [num_objects,]
        cosine_angle = sign * np.arccos( \
            np.dot(object_direction, np.array([-1.0, 0.0])))

        # differnce in angle between the head angle and sine_angle of head
        relative_angle = cosine_angle + self.head.angle

        # relative_angle should be within [-np.pi, np.pi]
        for i in range(relative_angle.shape[0]):
            if relative_angle[i] < -np.pi:
                k = 1
                while relative_angle[i] < (k + 1) * -np.pi:
                    k += 1
                relative_angle[i] = relative_angle[i] + 2 * np.pi * ((k + 1) // 2)

            elif relative_angle[i] > np.pi:
                k = 1
                while relative_angle[i] > (k + 1) * np.pi:
                    k += 1
                relative_angle[i] = relative_angle[i] - 2 * np.pi * ((k + 1) // 2)

        return relative_angle

    def _get_angular_velocity(self, prev_ang, current_ang):
        angle_velocity = current_ang - prev_ang
        angle_speed = np.absolute(angle_velocity)
        for i in range(angle_velocity.size):
            if angle_speed[i] > np.pi:
                angle_velocity[i] = 2 * np.pi - angle_velocity[i]
            elif angle_speed[i] < -np.pi:
                angle_velocity[i] = 2 * np.pi + angle_velocity[i]
            else:
                pass
        return angle_velocity

    """
    Defining Reward Functions
    """
    def _assign_reward_func(self, reward_code, max_offset = None):
        self.prev_angle = self._get_retinal(self.objects_position)
        if "motion_parallax" in reward_code:
            self.reward_function = self._motion_parallax
        elif "retinal_stabilization" in reward_code:
            self.reward_function = self._retinal_stabilization
        elif "fifty_fifty" in reward_code:
            self.reward_function = self._fifty_fifty
        else:
            raise ValueError("Unknown reward_code")

    def _motion_parallax(self):
        current_angle = self._get_retinal(self.objects_position)

        parallax_velocities = \
            self._get_angular_velocity(current_angle, self.prev_angle)

        reward = 0
        # sum of motion parallax magnitudes
        for i in range(parallax_velocities.size):
            for j in range(i, parallax_velocities.size):
                reward += np.abs(parallax_velocities[i] - parallax_velocities[j])
            # reward += parallax_velocities[i]
        return reward

    def _retinal_stabilization(self):
        reward = 0
        current_angle = self._get_retinal(self.objects_position)
        relative_speeds = \
            np.absolute(self._get_angular_velocity(current_angle, self.prev_angle))
        reward -= np.sum(relative_speeds)
        return reward

    def _fifty_fifty(self):
        reward = 0
        reward += self._retinal_stabilization()
        reward += self._motion_parallax()
        return reward

    def _get_obs(self):
        # (self.head{relative}, self.joints -> obs) operation
        obs = np.array(self.head.position) - np.array(self.body.position)
        obs = np.concatenate((obs, self.head.angle), axis = None)
        for i in range(len(self.joints)):
            obs = np.concatenate((obs, self.joints[i].angle), axis = None)
            obs = np.concatenate((obs, self.joints[i].speed), axis = None)
        obs = np.concatenate((obs, self.body.position[0]), axis = None)
        obs = np.float32(obs)
        assert self.observation_space.contains(obs)
        return obs

    def step(self, action):
        self.prev_angle = self._get_retinal(self.objects_position)
        # alter object
        self.objects_position += self.objects_velocity
        return super().step(action)

    def render(self, mode = "human"):
        from gym.envs.classic_control import rendering
        if self.viewer is None:
            self.render_objects_list = None
            self.render_objects_translate_list = None

        super().render(mode)
        # initialize object rendering pointers
        if self.render_objects_list is None:
            self.render_objects_list = []
            self.render_objects_translate_list = []
            for i in range(self.objects_position.shape[0]):
                object_render_instance = rendering.make_circle( \
                    radius=0.6,
                    res=30,
                    filled=True)
                object_render_instance_translate = rendering.Transform(
                    translation = VIEWPORT_SCALE * \
                        (self.objects_position[i] - self.camera_trans),
                    rotation = 0.0,
                    scale = VIEWPORT_SCALE * np.ones(2)
                )
                object_render_instance.add_attr(object_render_instance_translate)
                object_render_instance.set_color(0.0, 1.0, 0.0)
                self.render_objects_list.append(object_render_instance)
                self.render_objects_translate_list.append(object_render_instance_translate)
                self.viewer.add_geom(object_render_instance)

        # update object translation
        new_object_translate = VIEWPORT_SCALE * self.objects_position - self.camera_trans
        for i in range(self.objects_position.shape[0]):
            self.render_objects_translate_list[i].set_translation( \
                new_object_translate[i][0], new_object_translate[i][1])

        return self.viewer.render(return_rgb_array = mode == "rgb_array")
\end{lstlisting}


\end{document}
